\documentclass[12pt, letter]{article}

\usepackage{synttree}

\title{Lab 1}
\author{James You}

\begin{document}
\maketitle
\newpage
\section*{Question 1}
\subsection*{a)}
\synttree[S 
            [NP
            	[CN
            		[A [a]]
            		[NN [cat]]
            	]
            ] 
			[VP
				[CV
					[V [scratches]]
					[NP 
						[CN 
							[A [the]]
							[NN [baby]]
						]
					]
				]
				[PP 
					[PR [with]] 
					[CN 
						[A [a]] 
						[NN [nail]]
					]
				]
			]
		 ]
		
\synttree[S
			[NP
				[CN [A [a]] [NN [cat]]]
			]
			[VP
				[CV [V [scratches]]
					[NP [CN [A [the]] [NN [baby]]]
						[PP 
							[PR [with]]
							[CN [A [a]] [NN [nail]]]
						]
					]
				]
			] 
		 ]

\subsection*{b)}		 
In the sentence "a cat scratches the baby with a nail", it's ambiguous whether the cat is using a nail to scratch the baby or the baby possesses the nail and the cat is simply scratching the baby (without the nail). It is ambiguous because ownership of the nail is not context free, it is unknown whether the cat possesses the nail or the baby does.


\section*{Question 2}
\subsection*{a)}
	Let G = \{T, N, P, S\} be a grammar with T = \{a, b ,c ,d, /, *\}, and N = \{A, B, DIV, MUL, ID\} and productions P given as: \\
	\texttt{S -> A \\
			A -> B DIV A | B \\
			B -> ID MUL B | ID \\
	        DIV -> / \\
            MUL -> * \\
            ID -> a | b | c | d} \\ \\
    parse tree for "a * b / c * d" for part a)\\ \\
    \synttree[S [A [B [ID [a]] [MUL [*]] [B [ID [b]]]]
                   [DIV [/]]
                   [A [B [ID [c]] [MUL [*]] [B [ID [d]]]]]
                ]
    		 ]
	\subsection*{b)}
	Let G = \{T, N, P, S\} be a grammar with T = \{a, b ,c ,d, /, *\}, and N = \{A, B, DIV, MUL, ID\} and productions P given as: \\
	\texttt{S -> A \\
		A -> B MUL A | B \\
		B -> ID DIV B | ID \\
		DIV -> / \\
		MUL -> * \\
		ID -> a | b | c | d} \\ \\
	parse tree for "a * b / c * d" for part b)\\
	\synttree[S [A [B [ID [a]]]
	               [MUL [*]]
	               [A [B [ID [b]]
	                     [DIV [/]]
	                     [ID [c]]
	                  ]
	                  [MUL [*]]
	                  [A [B [ID [d]]]]
	               ] 
	            ]
			 ]
	\subsection*{c)}
	Let G = \{T, N, P, S\} be a grammar with T = \{a, b ,c ,d, /, *\}, and N = \{A, OP, ID\} and productions P given as: \\
	\texttt{S -> A \\
		A -> A OP ID | ID \\
		OP -> / | * \\
		ID -> a | b | c | d} \\ \\
	parse tree for "a * b / c * d" for part c)\\ \\
	\synttree[S [A [A [A [A [ID [a]]]
	                     [OP [*]]
	                     [ID [b]]
	                  ]
	                  [OP [/]]
	                  [ID [c]]
				   ]
			       [OP [*]]
			       [ID [d]]
			    ]
			 ]
	\subsection*{d)}
	Let G = \{T, N, P, S\} be a grammar with T = \{a, b ,c ,d, /, *\}, and N = \{A, OP, ID\} and productions P given as: \\ \ \\
	\texttt{S -> A \\
		A -> ID OP A | ID \\
		OP -> / | * \\
		ID -> a | b | c | d} \\ \\
	parse tree for "a * b / c * d" for part d)\\ \\
	\synttree[S [A [ID [a]]
	               [OP [*]]
	               [A [ID [b]]
	                  [OP [/]]
	                  [A [ID [c]]
	                     [OP [*]]
	                     [ID [d]] 
	                  ]
	               ]
	            ]
]
\section*{Question 3}

The inclusion $L(G) \subseteq \{a^n b c^n\}$ specifies all elements of $L(G)$ to be of the form $a^n b c^n$ for all $n \geq 0$. We start by proving inclusion on production $(2)$ such 
$ S \rightarrow A \Rightarrow b = a^0 b c^0 $ for $ n = 0 $. Then for production $ (3) $, if we assume the $A$ on the right side is of the form $a^n b c^n$ terminating through production $(2)$. The production $aAc$ will always produce $a^{n+1} b c^{n+1}$. Therefore $L(G) = \{a^n b c^n\}$.

Then, we must prove that $\{a^n b c^n\}$ for $n \geq 0$ can be generated by G. We specify the base case $a^0 b c^0 $ can be generated through $ A \Rightarrow b $. The case $a^1 b c^1 $ can be generated by $ A \Rightarrow aAc $ s.t. the right A terminates in $ A \rightarrow b $. Given the cases for $n = 0$ and $n = 1$. We can show that  $a^{n+1} b c^{n+1}$ can be generated by $ A \Rightarrow aAc \Rightarrow^*aa^nbc^nc = a^{n+1} b c^{n+1} $. Therefore $L(G) \supseteq \{a^n b c^n\ |\ n \geq 0\} $ and $L(G) = \{a^n b c^n\}$. 

\end{document}